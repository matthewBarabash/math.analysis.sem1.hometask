\documentclass[a4paper,12pt]{report} 

\usepackage{cmap}				
\usepackage{mathtext} 				
\usepackage[T2A]{fontenc}			
\usepackage[utf8]{inputenc}			
\usepackage[english,russian]{babel}	

\usepackage{amsmath,amsfonts,amssymb,amsthm,mathtools} % AMS
\usepackage{icomma}

% Показывать номера только у тех формул, на которые есть \eqref{} в тексте.
\mathtoolsset{showonlyrefs=true} 

%% Свои команды
\DeclareMathOperator{\sgn}{\mathop{sgn}}

%% Перенос знаков в формулах (по Львовскому)
\newcommand*{\hm}[1]{#1\nobreak\discretionary{}
{\hbox{$\mathsurround=0pt #1$}}{}}

%%% Заголовок
\author{Матвей Барабашкин\\vk.com/matthew\_barabash}
\title{Матанчик}
\date{20 июля 2017}

\begin{document}

\maketitle

\chapter*{Первое задание.}

\section*{Неделя первая.\\ Действительные и комплексные числа.}

\subsection*{\S3.\\}

\paragraph{Задача №1(4)*.}
\subparagraph{(Записать $ a=1,3(18) $ в виде $ p/q $).}
Т.к. $ 1000a=1318,(18) $, $ 10a\hm=13,(18) $, то 
$ 990a=1305 $. Откуда получаем, что $ a=\dfrac{1305}{990}=\dfrac{29}{22} $

\paragraph{Задача №2*.}\label{sqrt2}
\subparagraph{(Док-ть: $ \sqrt{2} $~--- иррациональное).}
Предположим противное.
Пусть $ \sqrt{2} $~--- рациональное, тогда
его можно представить в виде $ (\frac{p}{q})^2=2 $, где
$ p $ и $ q $ целые числа, не имеющие общих множителей.
Т.е. будет считать дробь $ p/q $ несократимой. Тогда
из $ p^2=2q^2 $ следует, что $ p $~--- \textbf{четно} или
то же самое $ p=2k $ ($ k $ целое). Следовательно,
$ q $ --- \textbf{нечетно}. 
Подставляя $ p=2k $ в исходное соотношение, находим
$ q^2=2k^2 $. Откуда $ q $~--- \textbf{четно}. 
Т.е. противоречие (т.к. дробь несократимая). Следовательно,
$ \sqrt{2} $~--- иррациональное.

\paragraph{(Док-ть: $ \sqrt{3} $--- иррациональное).}
Предположим противное.
Пусть $ \sqrt{3} $~--- рациональное, тогда
его можно представить в виде $ (\frac{p}{q})^2=3 $, где
$ p $ и $ q $ целые числа, не имеющие общих множителей.
Т.е. будет считать дробь $ p/q $ несократимой. Тогда
из $ p^2=3q^2 $ следует, что $ p^2 $ делится нацело на 3
и, следовательно, $ p $ тоже делится на 3.
Другими словами, $ p=3k $ ($ k $ целое).
Подставляя $ p=3k $ в исходное соотношение, находим
$ q^2=3k^2 $. Откуда аналогичным образом $ q $ так же делится на 3. 
Т.е. противоречие (т.к. дробь несократимая). Следовательно,
$ \sqrt{3} $~--- иррациональное.

\paragraph{(Док-ть: $ \sqrt{2/3} $ --- иррациональное).}
Из предыдущих пунктов заметим, что $ \sqrt{2} $ и
$ \sqrt{3} $~--- есть числа иррациональные.
$ \sqrt{\dfrac{2}{3}}=\dfrac{\sqrt{6}}{3}=\dfrac{1}{3}\sqrt{6} $.
Теперь докажем, что $ \sqrt{6} $~--- иррациональное. 
Предположим противное, тогда $ \sqrt{6}=\dfrac{p}{q} $.
Возводя в квадрат и домнажая на $ q^2 $ имеем: $$ 6q^2=p^2 
\Rightarrow p=2k \text{ (k целое).}$$ 
$ \sqrt{6}=\dfrac{2k}{q} $ или 
$ q\sqrt{6}=2k \Rightarrow q=2k\text{ (k целое).}$  
Имеем, что $ p $ и $ q $ делятся на 2, т.е. получили противречие.
Следовательно $ \sqrt{2/3} $~--- иррациональное.

\paragraph{(Док-ть: $ \sqrt{2} + \sqrt{3} $~--- иррациональное).}
Предположим противное.
Пусть $ \sqrt{2} + \sqrt{3} $~--- рациональное, тогда
$ \sqrt{2} + \sqrt{3} = p/q $. Возводя в квадрат обе части получим: 
$ 2 + 2\sqrt{6} + 3 = 
5 + 2\sqrt{6} = (p/q)^2 $ или 
$ \sqrt{6}= \dfrac{(p/q)^2 - 5}{2} $.  
Слева от равенства стоит $ \sqrt{6} $~--- \textbf{иррациональное} 
число (см. предыдущий пункт), а справа от равенства 
стоит \textbf{рацональное} число. Т.е. противоречие.
Следовательно, $ \sqrt{2} + \sqrt{3} $~--- иррациональное.

\paragraph{Задача 3*.}
\subparagraph{(Может ли сумма двух иррациональных чисел быть рациональным числом?).} 
В качестве примера возьмем $ \sqrt{2} $~--- иррациональное число.
Возьмем к нему обратный по сложению элемент: $ -\sqrt{2} $. Тогда,
$ \sqrt{2} + (-\sqrt{2}) = 0 $. Т.к. 0~--- рациональное число, то 
ответ: <<Да, может>>.

\subsection*{\S5.\\}

\paragraph{Задача №4(5).}

\begin{align*}
	&\dfrac{(1+2i)^2-(1-i)^3}{(3+2i)^3-(2+i)^2}=
	\dfrac{1+4i-4-(1-3i-3+i)}{27+54i-36-8i-(4+4i-1)}=\\
	&=\dfrac{-3+4i-(1-3i-3+i)}{-9+46i-(4+4i-1)}=
	\dfrac{-3+4i-1+3i+3-1i}{-9+46i-4-4i+1}=\\
	&=\dfrac{-1+6i}{-12+42i}=
	\dfrac{(-1+6i)(-12-42i)}{(-12+42i)(-12-42i)}=
	\dfrac{12+42i-72i+252}{144+1764}=\\
	&=\dfrac{264-30i}{1908}=
	\fbox{$ \dfrac{22}{159}+\dfrac{5}{318}i $} 
\end{align*}

\paragraph{Задача №25(4).}
\subparagraph{(Предстваить комплексное число z в тригонометрической форме).}

\begin{align*}
	&z=1+\cos{\dfrac{10\pi}{9}}+i\sin{\dfrac{10\pi}{9}}=
	\left|
		\begin{aligned}
			&\cos{\dfrac{10\pi}{9}}=2\cos^2{\dfrac{5\pi}{9}}-1\\
			&\sin{\dfrac{10\pi}{9}}=
			2\sin{\dfrac{5\pi}{9}}\cos{\dfrac{5\pi}{9}}
		\end{aligned}
	\right|=\\
	&=1+2\cos^2{\dfrac{5\pi}{9}}-1+
	2i\sin{\dfrac{5\pi}{9}}\cos{\dfrac{5\pi}{9}}=
	2\cos^2{\dfrac{5\pi}{9}}+
	2i\sin{\dfrac{5\pi}{9}}\cos{\dfrac{5\pi}{9}}=\\
	&=2\cos{\dfrac{5\pi}{9}}
	\left(
		\cos{\dfrac{5\pi}{9}}+i\sin{\dfrac{5\pi}{9}}
	\right)=
	-2\cos{\dfrac{5\pi}{9}}
	\left(
	\cos{\dfrac{14\pi}{9}}+i\sin{\dfrac{14\pi}{9}}
	\right)
\end{align*}
 
\paragraph{Задача №15(2, 5).}
\subparagraph{(Найти множество точек комплексной плоскости, 
	удовлетворяющих условию: $ |z-i|<|z+i| $).} 
Т.к. $ z $ - есть число комплексное, запишем его в алгебраической форме
$ z=a+bi $, тогда:
\begin{align*}
	&|x+iy-i|<|x+iy+i|\\
	&|x+i(y-1)|<|x+i(y+1)|\\
	&\sqrt{x^2+(y-1)^2}<\sqrt{+(y+1)^2}\\
	&x^2+(y-1)^2<x^2+(y+1)^2\\
\end{align*}

\begin{align*}
	&(y-1)^2-(y+1)^2<0\\
	&(y-1+y+1)(y-1-y-1)<0\\
	&2y(-2)<0\\
	&y>0	
\end{align*}

В итоге, получаем условие $ y>0 $ , что представляет собой верхнюю
полуплоскость без оси абсцисс.

\subparagraph{(Найти множество точек комплексной плоскости, 
	удовлетворяющих условию: $ |z-2|+|z+2|=26 $).\\\\}

\textbf{Первый способ}.
Условие $ |z-2|+|z+2|=26 $ означает, что сумма расстояний от точек
$ -2 $ и 2 равно всегда одному и тому же числу 26. 
Из этого следует,что мы имеем эллипс с фокусами $ F_1=-2 $ и $ F_2=2 $.
Большую полуось находим из равенства $ 2a=26 \Rightarrow a=13 $.\\

\textbf{Второй способ}.
Т.к. $ z $ - есть число комплексное, запишем его в алгебраической форме
$ z=a+bi $, тогда:
\begin{align*}
	&|a+bi-2|+|a+bi+2|=26\\
	&|(a-2)+bi|+|(a+2)+bi|=26\\	
	&\sqrt{(a-2)^2+b^2}+\sqrt{(a+2)^2+b^2}=26\\
	&\sqrt{(a-2)^2+b^2}=26-\sqrt{(a+2)^2+b^2}\\
	&(a-2)^2+b^2=26^2-52\sqrt{(a+2)^2+b^2}+(a+2)^2+b^2\\
	&a^2-4a+4=26^2-52\sqrt{(a+2)^2+b^2}+a^2+4a+4\\
	&-4a=26^2-52\sqrt{(a+2)^2+b^2}+4a\\
	&52\sqrt{(a+2)^2+b^2}=26^2+8a\\
	&13\sqrt{(a+2)^2+b^2}=13^2+2a\\
	&13^2(a^2+4a+4+b^2)=13^4+4\times13^2a+4a^2\\
	&13^2a^2+4\times 13^2a+4\times 13^2+13^2b=13^4+4\times13^2a+4a^2\\
	&13^2a^2+4\times 13^2+4\times 13^2+13^2b=13^4+4a^2\\
	&165a^2+169b^2=13^4-4\times 13^2=27885\\
	&\dfrac{a^2}{13^2}+\dfrac{b^2}{165}=1
\end{align*}
 
Полученное каноническое уравнение задает еллипс, 
большая полуось которого равна 13. 
Фокусы $ F_1(c, 0) $ и $ F_1(-c, 0) $ находим из соотношения
$ с=\sqrt{a^2-b^2}=\sqrt{169-165}=\sqrt{4}=2 $. 
Откуда $ F_1=(2, 0), F_2=(-2, 0) $.

\paragraph{Задача №19(3).}
\subparagraph{(Найти множество точек комплексной плоскости,
если один из аргументов $ \varphi $ удовлетворяет неравенствам 
$ 2\pi<\varphi<3\pi $).\\}

Угол $ \varphi $, удовлетворяющий неравенствам  $ 2\pi<\varphi<3\pi $ 
означает, что аргумент комплесного числа $ Arg(z) $ 
находится в пределах $ 2\pi<Arg(z)\hm<3\pi $. 
Т.к. $ Arg(z)=arg(z) + 2\pi k, \text{где } 0<arg(z)<2\pi, k \in \mathbb{Z}$,
то $ 0\hm<arg(z)<\pi $, т.е. множество комплексных точек лежит в первой и второй
четвертях, не влючая ось абсцисс, т.е. полуплоскость $ y>0 $.



\paragraph{Задача №30(1).}
\subparagraph{(Записать комплксное число $ z $ в алгебраической форме).}

\begin{align*}
	z=
	\left(
		\dfrac{i}{2}-\dfrac{\sqrt{3}}{2}	
	\right)^{12}=
	\left(	
		e^\frac{5\pi}{6}
	\right)^{12}=
	e^{10\pi}=e^0=1
\end{align*}

\paragraph{Задача №32(2).}
\subparagraph{(Найти все корни уравнения).}

\begin{align*}
	&z^3=8i\\
	&z=2\sqrt[3]{i}\\
	&z=2e^{(\pi /2+2\pi k)^{1/3}}
	\text{, } k \in \mathbb{Z}\\
	&z=2e^{(\pi/6+2\pi k/3)}
	\text{, }k=0,1,2\\
	&k=0: 
	\text{ }
	z_1=2e^{\pi /6}=\sqrt{3}+i\\
	&k=1: 
	\text{ }
	z_1=2e^{\pi /6}=-\sqrt{3}+i\\
	&k=12: 
	\text{ }
	z_1=2e^{\pi /6}=-2i\\[44pt]
\end{align*}

\paragraph{Задача №34(3).}
\subparagraph{(Записать в показательной и алгебраической форме).}

\begin{align*}	
	&z=(\sqrt{3}-i)^6=
	64
	\left(
		\dfrac{\sqrt{3}}{2}-\dfrac{i}{2}
	\right)^6=
	64e^{11i\pi}=64e^{i\pi}~\text{--- показательная форма.}\\
	&z=-64~\text{--- алгебраическая форма.}
\end{align*}

\section*{Неделя вторая.\\ Последовательности.}




\end{document} % конец документа
