\documentclass[a4paper,12pt]{report} 

\usepackage{cmap}				
\usepackage{mathtext} 				
\usepackage[T2A]{fontenc}			
\usepackage[utf8]{inputenc}			
\usepackage[english,russian]{babel}	

\usepackage{amsmath,amsfonts,amssymb,amsthm,mathtools} % AMS
\usepackage{icomma}

% Показывать номера только у тех формул, на которые есть \eqref{} в тексте.
\mathtoolsset{showonlyrefs=true} 

%% Свои команды
\DeclareMathOperator{\sgn}{\mathop{sgn}}

%% Перенос знаков в формулах (по Львовскому)
\newcommand*{\hm}[1]{#1\nobreak\discretionary{}
{\hbox{$\mathsurround=0pt #1$}}{}}

%%% Заголовок
\author{Матвей Барабашкин\\vk.com/matthew\_barabash}
\title{Матанчик}
\date{20 июля 2017}

\begin{document}

\maketitle

\chapter{Первое задание.}

\section{Неделя первая.\\ Действительные и комплексные числа.}

\subsection*{\S3.\\}

\paragraph{Задача №1(4)*}
\subparagraph{(Записать $ a=3,4(31) $ в виде $ p/q $).}
Т.к. $ 1000a=3431,(31) $, $ 10a\hm=34,(31) $, то 
$ 990a=3397 $. Откуда получаем, что $ a=\frac{3397}{990} $

\paragraph{Задача №2*}\label{sqrt2}
\subparagraph{(Док-ть: $ \sqrt{2} $~--- иррациональное).}
Предположим противное.
Пусть $ \sqrt{2} $~--- рациональное, тогда
его можно представить в виде $ (\frac{p}{q})^2=2 $, где
$ p $ и $ q $ целые числа, не имеющие общих множителей.
Т.е. будет считать дробь $ p/q $ несократимой. Тогда
из $ p^2=2q^2 $ следует, что $ p $~--- \textbf{четно} или
то же самое $ p=2k $ ($ k $ целое). Следовательно,
$ q $ --- \textbf{нечетно}. 
Подставляя $ p=2k $ в исходное соотношение, находим
$ q^2=2k^2 $. Откуда $ q $~--- \textbf{четно}. 
Т.е. противоречие (т.к. дробь несократимая). Следовательно,
$ \sqrt{2} $~--- иррациональное.

\paragraph{(Док-ть: $ \sqrt{3} $--- иррациональное).}
Предположим противное.
Пусть $ \sqrt{3} $~--- рациональное, тогда
его можно представить в виде $ (\frac{p}{q})^2=3 $, где
$ p $ и $ q $ целые числа, не имеющие общих множителей.
Т.е. будет считать дробь $ p/q $ несократимой. Тогда
из $ p^2=3q^2 $ следует, что $ p^2 $ делится нацело на 3
и, следовательно, $ p $ тоже делится на 3.
Другими словами, $ p=3k $ ($ k $ целое).
Подставляя $ p=3k $ в исходное соотношение, находим
$ q^2=3k^2 $. Откуда аналогичным образом $ q $ так же делится на 3. 
Т.е. противоречие (т.к. дробь несократимая). Следовательно,
$ \sqrt{3} $~--- иррациональное.

\paragraph{(Док-ть: $ \sqrt{2/3} $ --- иррациональное).}
Из предыдущих пунктов заметим, что $ \sqrt{2} $ и
$ \sqrt{3} $~--- есть числа иррациональные.
$ \sqrt{\dfrac{2}{3}}=\dfrac{\sqrt{6}}{3}=\dfrac{1}{3}\sqrt{6} $.
Теперь докажем, что $ \sqrt{6} $~--- иррациональное. 
Предположим противное, тогда $ \sqrt{6}=\dfrac{p}{q} $.
Возводя в квадрат и домнажая на $ q^2 $ имеем: $$ 6q^2=p^2 
\Rightarrow p=2k \text{ (k целое).}$$ 
$ \sqrt{6}=\dfrac{2k}{q} $ или $ q\sqrt{6}=2k \Rightarrow q=2k\text{ (k целое).}$  Имеем, что $ p $ и $ q $ делятся на 2, т.е. получили противречие. Следовательно $ \sqrt{2/3} $~--- иррациональное.

\paragraph{(Док-ть: $ \sqrt{2} + \sqrt{3} $~--- иррациональное).}
Предположим противное.
Пусть $ \sqrt{2} + \sqrt{3} $~--- рациональное, тогда
$ \sqrt{2} + \sqrt{3} = p/q $. Возводя в квадрат обе части получим: 
$ 2 + 2\sqrt{6} + 3 = 
5 + 2\sqrt{6} = (p/q)^2 $ или 
$ \sqrt{6}= \dfrac{(p/q)^2 - 5}{2} $.  
Слева от равенства стоит $ \sqrt{6} $~--- \textbf{иррациональное} 
число (см. предыдущий пункт), а справа от равенства 
стоит \textbf{рацональное} число. Т.е. противоречие.
Следовательно, $ \sqrt{2} + \sqrt{3} $~--- иррациональное.

\paragraph{Задача 3*}
\subparagraph{(Может ли сумма двух иррациональных чисел быть рациональным числом?).} 
В качестве примера возьмем $ \sqrt{2} $~--- иррациональное число.
Возьмем к нему обратный по сложению элемент: $ -\sqrt{2} $. Тогда,
$ \sqrt{2} + (-\sqrt{2}) = 0 $. Т.к. 0~--- рациональное число, то 
ответ: <<Да, может>>.

\subsection{\S4.\\}

\paragraph{Задача №4(5)}

\begin{align*}
	&\dfrac{(1+2i)^2-(1-i)^3}{(3+2i)^3-(2+i)^2}=
	\dfrac{1+4i-4-(1-3i-3+i)}{27+54i-36-8i-(4+4i-1)}=\\
	&=\dfrac{-3+4i-(1-3i-3+i)}{-9+46i-(4+4i-1)}=
	\dfrac{-3+4i-1+3i+3-1i}{-9+46i-4-4i+1}=\\
	&=\dfrac{-1+6i}{-12+42i}=
	\dfrac{(-1+6i)(-12-42i)}{(-12+42i)(-12-42i)}=
	\dfrac{12+42i-72i+252}{144+1764}=\\
	&=\dfrac{264-30i}{1908}=
	\fbox{$ \dfrac{22}{159}+\dfrac{5}{318}i $} 
\end{align*}

\paragraph{Задача №13}
\subparagraph{(Найти модуль комплексного числа z).}

\begin{align*}
	&z=1+\cos{\dfrac{8\pi}{7}}+i\sin{\dfrac{8\pi}{7}}=	
	1+\cos{(\pi+\dfrac{\pi}{7})}+i\sin{(\pi+\dfrac{\pi}{7})}=	
	1-\cos\pi/7-i\sin\pi/7;\\
	&|z|=\sqrt{\left(1-\cos{\dfrac{\pi}{7}}\right)^2+sin^2{\dfrac{\pi}{7}}}=
	\sqrt{1-2\cos{\dfrac{\pi}{7}}+
	\cos^2{\dfrac{\pi}{7}}+\sin^2\dfrac{\pi}{7}}=
	\sqrt{2-2\cos{\dfrac{\pi}{7}}}=\\
	&=\sqrt{2\left(2\sin^2{\dfrac{\pi}{14}}\right)}=
	\fbox{$ 2\sin{\dfrac{\pi}{14}} $}
\end{align*}
 
\paragraph{Задача №15}
\subparagraph{(Найти множество точек комплексной плоскости, удовлетворяющих условию: $ |z-i|<|z+i| $).} 
Данное условие задает множество точек, расстояние от которых 
до точки $ i $ меньше, чем до $ -i $. Имеем верхнюю полуплоскость $ y>0 $.

\subparagraph{(Найти множество точек комплексной плоскости, удовлетворяющих условию: $ |z-2|^2+|z+2|^2=26 $).\\}
Т.к. $ z $ - есть число комплексное, запишем его в алгебраической форме
$ z=a+bi $, тогда:
\begin{align*}
	&|a+bi-2|^2+|a+bi+2|=26\\
	&|(a-2)+bi|^2+|(a+2)+bi|=26\\	
	&a^2-4a+4+b^2+a^2+4a+4+b^2=26
\end{align*}


\begin{align*}
	&2a^2+2b^2=18\\
	&a^2+b^2=9\\
	&\sqrt{a^2+b^2}=\pm 3
\end{align*}



\end{document} % конец документа
